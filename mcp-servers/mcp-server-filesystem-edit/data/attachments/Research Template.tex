%
% Template
%

\documentclass[10pt,a4paper,twocolumn]{article}

% Essential packages
\usepackage[utf8]{inputenc}
\usepackage[T1]{fontenc}
% Add improved fonts
\usepackage{lmodern} % Latin Modern fonts - improved version of Computer Modern
% Alternative: \usepackage{times} for Times New Roman-like font

\usepackage{amsmath,amssymb,amsfonts}
\usepackage{graphicx}
\usepackage[normalem]{ulem}
\usepackage{hyperref}
\usepackage{booktabs}
\usepackage{algorithm}
\usepackage{algorithmic}
\usepackage[font=small,labelfont=bf]{caption}

% Carefully add microtype back with options that avoid expansion
\usepackage[activate={true,nocompatibility},final,tracking=true,kerning=true,spacing=true,factor=1100]{microtype}
% Disable problematic font expansion
\microtypesetup{expansion=false}

% Page layout
\usepackage[top=2cm, bottom=2cm, left=2cm, right=2cm]{geometry}
\setlength{\columnsep}{0.5cm}

% Title and author information
\title{Your Paper Title}
\author{
  First Author\\
  Institution\\
  \texttt{first.author@example.com}
  \and
  Second Author\\
  Institution\\
  \texttt{second.author@example.com}
}
\date{\today}

\begin{document}

\maketitle

\begin{abstract}
This document provides a template for academic papers similar to conference formatting requirements. The abstract should be a concise summary of your work, typically 150-250 words. It should describe the problem addressed, the approach taken, and the key results or contributions.
\end{abstract}

\section{Introduction}
\label{sec:introduction}

The introduction section establishes the context of your work. It typically includes motivation for the research, a brief overview of related work, and a statement of the specific contributions of your paper (Smith et al., 2020).

\section{Related Work}
\label{sec:related}

The related work section situates your research within the context of existing literature. It should be organized thematically, highlighting both the accomplishments and limitations of prior approaches (Jones et al., 2019).

\section{Methodology}
\label{sec:method}

This section describes your approach in detail. It should be comprehensive enough that a reader could replicate your work.

\subsection{Problem Formulation}
You might define your problem mathematically:

\begin{equation}
\min_{w} \frac{1}{n} \sum_{i=1}^{n} L(f(x_i; w), y_i) + \lambda R(w)
\end{equation}

Where $L$ is a loss function and $R$ is a regularization term.

\subsection{Algorithm}
\label{subsec:algorithm}

You can present pseudocode for your algorithm:

\begin{algorithm}
\caption{Example Algorithm}
\label{alg:example}
\begin{algorithmic}
\STATE \textbf{Input:} Data $X$, parameters $\theta$
\STATE \textbf{Output:} Result $Y$
\STATE Initialize $Y \gets \emptyset$
\FOR{each $x \in X$}
    \STATE $y \gets \text{process}(x, \theta)$
    \STATE $Y \gets Y \cup \{y\}$
\ENDFOR
\RETURN $Y$
\end{algorithmic}
\end{algorithm}

\section{Experiments}
\label{sec:experiments}

\subsection{Datasets}
Describe the datasets used in your evaluation.

\subsection{Implementation Details}
Provide details about your implementation, including hyperparameters, software, and hardware used.

\subsection{Results}

Tables are useful for presenting quantitative results:

\begin{table}
\caption{Comparison of Methods}
\label{tab:results}
\centering
\begin{tabular}{lccc}
\toprule
Method & Accuracy & Precision & Recall \\
\midrule
Baseline & 0.82 & 0.80 & 0.85 \\
Our Method & \textbf{0.91} & \textbf{0.89} & \textbf{0.92} \\
\bottomrule
\end{tabular}
\end{table}

Figures help visualize your results:

\begin{figure}
\centering
% Replace with your actual figure
\rule{5cm}{4cm} % This creates a black rectangle as a placeholder
\caption{Example visualization of results.}
\label{fig:results}
\end{figure}

\section{Conclusion}
\label{sec:conclusion}

Summarize your contributions and discuss potential directions for future work.

\section*{Acknowledgments}
Acknowledge funding sources and individuals who provided assistance.

\section*{References}
% Simple manually formatted references instead of using natbib
\begin{enumerate}
\item Smith, J., Johnson, A., \& Williams, B. (2020). Example paper title. \textit{Journal of Important Research}, 42(1), 123-456.
\item Jones, M., Brown, L., \& Davis, R. (2019). A survey of relevant techniques. In \textit{Proceedings of Example Conference} (pp. 78-89).
\end{enumerate}

\end{document}