\documentclass[12pt,a4paper]{article}

% Essential packages
\usepackage[utf8]{inputenc}
\usepackage[T1]{fontenc}
\usepackage{mathptmx}         % Times New Roman font
\usepackage{graphicx}         % For images
\usepackage{amsmath,amssymb}  % Math symbols
\usepackage[hidelinks]{hyperref}  % For clickable references
\usepackage[left=2.5cm,right=2.5cm,top=2.5cm,bottom=2.5cm]{geometry}  % Margins
\usepackage{natbib}           % For citations
\usepackage{setspace}         % For line spacing
\usepackage{booktabs}         % For professional tables
\usepackage{microtype}        % Typography improvements

% Document settings
\setlength{\parindent}{1.5em}
\setlength{\parskip}{0.5em}
\onehalfspacing

% Title information
\title{Paper Title: Subtitle}
\author{Author Name\\
\small{Department, Institution}\\
\small{\texttt{email@example.com}}
}
\date{\today}

\begin{document}

\maketitle

\begin{abstract}
This is the abstract of your paper. It should be a concise summary of the main points, methods, findings, and conclusions. Generally, an abstract should be between 150-250 words. Write this section last after completing your paper to ensure it accurately reflects the content.
\end{abstract}

\section{Introduction}
The introduction should provide context for your research, state the problem being addressed, and outline the purpose and significance of your work. End with a clear statement of your research question or hypothesis and a brief overview of your approach.

\section{Related Work}


This section should review relevant literature and establish how your work builds on or differs from previous research. Organize this section thematically rather than chronologically.



\section{Methodology}
Describe your research methods in sufficient detail for others to replicate your work. Include information about data collection, experimental setup, analytical techniques, or theoretical frameworks as appropriate.

\section{Results}
Present your findings without interpretation. Use subsections for clarity if needed:

\subsection{First Result Category}
Results can be presented in text, tables, or figures as appropriate. For example:

\begin{table}[h]
\centering
\caption{Sample Table of Results}
\begin{tabular}{@{}lcc@{}}
\toprule
Factor & Value & Significance \\
\midrule
Factor 1 & 12.3 & p < 0.05 \\
Factor 2 & 45.6 & p < 0.01 \\
Factor 3 & 78.9 & n.s. \\
\bottomrule
\end{tabular}
\end{table}

\subsection{Second Result Category}


Some content here

\subsubsection{Deeper analysis}
This is an even deeper section

\section{Discussion}
Interpret your results in the context of your research question and the existing literature. Address any limitations of your study and suggest implications for theory or practice.

\section{Conclusion}
Summarize your main findings and their significance. Suggest directions for future research.

\section*{Acknowledgments}
Thank individuals or organizations who contributed to this work but are not listed as authors. Mention any funding sources.

\bibliographystyle{plainnat}
% The bibliography file should be named references.bib
\bibliography{references}

\appendix
\section{Supplementary Material}
Include any additional information that supports your paper but would disrupt the flow of the main text.

\end{document}